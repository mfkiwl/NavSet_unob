% ********************************** CHAPTER 4 source ***********************************************
\section{Generátor rádiového signálu}


\marginpar{\textcolor{txt_blue}{up-convertor}} 
V digitálním rádiovém systému slouží blok "Up-Convertor" k transformaci připraveného signálu v základním pásmu (baseband) do kmitočtového pásma vhodného k rádiovému přenosu. řekněme, že daný krok lze nazvat i "modulací nosné".

V tomto projektu bude vytvořen Up-convertor pro účely simulace. Ve vlastní hardwarové implementaci slouží k přeměně samotné rádio, čili fyzicky RF-frontend.


%%%%%%%%%%%%%%%%%%%%%%%%%%%%%%%%%%%%%%%% Diff Equation %%%%%%%%%%%%%%%%%%%%%%%%%%%%%%%%%%%%%%%%%%%%
\subsection{Realizace Up-convertoru - Python}
\marginpar{\textcolor{txt_blue}{Rozbor}} 
Let's start with a time-domain first to derive the differential equation from a \textsl{free-body diagram}. We will use the second Newton's law, setting the mass $M=0$.


%%%%%%%%%%%%%%%%%%%%%%%%%%%%%%%%%%%%%%%%%% Time Response %%%%%%%%%%%%%%%%%%%%%%%%%%%%%%%%%%%%%%%%%%%%
\subsection{Použití bloku USRP v GNU Radio}
We are required to derive both natural and step response.



