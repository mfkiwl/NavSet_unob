% ********************************** CHAPTER 4 source ***********************************************
\section{Generátor rádiového signálu}

Z úvodní části tohoto textu a z obrázku \ref{fig_block_Tx} je patrno několik stupňů -- fází, ze kterých se celý proces generování rádiového signálu skládá. Uvažujme, že vstupem je posloupnost, viz rovnice (\ref{eq:bseq}). V první řadě jde o formování požadovaných n-tic bitů přenášené bitové posloupnosti $b(n)$, podle zvoleného počtu stavů modulace $M$. Každá z n-tic bitů definuje hodnotu patřičného symbolu. Hodnotu symbolu je potřeba "namapovat" na patřičnou podobu složek signálu základního pásma a poté složky I a Q, tedy $s_m^I(t)$ a $s_m^Q(t)$, podle rovnic (\ref{eq:waveformsI}) a (\ref{eq:waveformsQ}) vygenerovat.



\marginpar{\textcolor{txt_blue}{up-convertor}} 
V digitálním rádiovém systému slouží blok "Up-Convertor" k transformaci připraveného signálu v základním pásmu (baseband) do kmitočtového pásma vhodného k rádiovému přenosu. řekněme, že daný krok lze nazvat i "modulací nosné".

V tomto projektu bude vytvořen Up-convertor pro účely simulace. Ve vlastní hardwarové implementaci slouží k přeměně samotné rádio, čili fyzicky RF-frontend.


%%%%%%%%%%%%%%%%%%%%%%%%%%%%%%%%%%%%%%%% Diff Equation %%%%%%%%%%%%%%%%%%%%%%%%%%%%%%%%%%%%%%%%%%%%
\subsection{Realizace Up-convertoru - Python}
\marginpar{\textcolor{txt_blue}{Rozbor}} 
Let's start with a time-domain first to derive the differential equation from a \textsl{free-body diagram}. We will use the second Newton's law, setting the mass $M=0$.


\subsection{Způsob generování}

\subsubsection {Přímý generátor}

\subsubsection {Root raised -- cosine filtr}


%%%%%%%%%%%%%%%%%%%%%%%%%%%%%%%%%%%%%%%%%% Time Response %%%%%%%%%%%%%%%%%%%%%%%%%%%%%%%%%%%%%%%%%%%%
\subsection{Použití bloku USRP v GNU Radio}
We are required to derive both natural and step response.


\lstinputlisting[language=Python, caption={Mapování bitové posloupnosti} ,label=lst_recttr]{./ch_04/src/constallation_mappers.py}
